\usepackage{xcolor}
\definecolor{darkblue}{RGB}{0,37,70}
\definecolor{darkred}{RGB}{70,0,38}
\definecolor{darkorange}{RGB}{231,129,0}
\definecolor{darkbrown}{RGB}{108,60,0}
\definecolor{darkgreen}{RGB}{5,130,0}

\usetheme[numbering=fraction,progressbar=head,background=light,block=fill]{metropolis}

\usepackage{etoolbox}

\setbeamercolor{block title}{use=structure,fg=structure.fg,bg=structure.fg!20!bg}
\setbeamercolor{block body}{parent=normal text,use=block title,bg=block title.bg!50!bg}

\setbeamercolor{block title example}{use=example text,fg=example text.fg,bg=example text.fg!20!bg}
\setbeamercolor{block body example}{parent=normal text,use=block title example,bg=block title example.bg!50!bg}

\BeforeBeginEnvironment{theorem}{
    \setbeamercolor{block title}{fg=normal text.fg,bg=structure.fg!10!bg}
    \setbeamercolor{block body}{fg=normal text.fg, bg=white!97!black}
}
\AfterEndEnvironment{theorem}{
 \setbeamercolor{block title}{use=structure,fg=structure.fg,bg=structure.fg!20!bg}
 \setbeamercolor{block body}{parent=normal text,use=block title,bg=block title.bg!50!bg, fg=normal text.fg}
}

\setbeamercolor{normal text}{%
	fg=darkblue,
	bg=white
}

\setbeamercolor{title page text}{
	fg=white,
	bg=transparent}

\usepackage{pgfpages}

\usepackage[numbers,square,sort&compress]{natbib}
\usepackage[american]{babel}
\setbeamertemplate{bibliography item}{\insertbiblabel}

% \usepackage{booktabs}
\usepackage{algorithmic}

% Definitions
\newcommand{\myparallel}{\textcolor{darkbrown}{parallel}}
\newcommand{\mysequential}{\textcolor{darkgreen}{sequential}}

\newcommand{\obj}{\mathcal{O}}
\newcommand{\kl}{D_{KL}}
\newcommand{\proxy}{\mathcal{P}}
\newcommand{\R}{\mathbb{R}}
\newcommand{\orth}{\mathcal{O}}
\newcommand{\I}{\mathbb{1}}
% \newcommand{\C}{\mathbb{C}}
\newcommand{\E}{\mathbb{E}}
\newcommand{\lrelu}{\text{LeakyReLU}}
\newcommand{\bias}{\text{Bias}}
\newcommand{\cnn}{\text{CNN}}
\newcommand{\eps}{\varepsilon}
\newcommand{\conv}{\text{conv}_{3d}}
\newcommand{\convtwo}{\text{conv}_{2d}}
\newcommand{\dft}{\text{DFT}_{3d}} % default is 3, add 2 also. 
\newcommand{\idft}{\text{DFT}^{-1}_{3d}}
\newcommand{\channels}{\text{\# channels}}
\DeclareMathOperator*{\argmax}{\arg{}\,\max}
\DeclareMathOperator*{\argmin}{\arg{}\,\min}
% end definitions

% \setbeameroption{show notes on second screen}
% \makeatletter
% \def\beamer@framenotesbegin{% at beginning of slide
% \usebeamercolor[fg]{normal text}
% \gdef\beamer@noteitems{}%
% \gdef\beamer@notes{}%
% }
% \makeatother
