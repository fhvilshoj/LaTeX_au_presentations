\begin{frame}{Explanations}
	Here we focus on \alert{explainability}, characterized by
	\vspace{1em}

	\begin{quote}
		An active characteristic of a model, denoting any action or procedure taken by a model with the intent of clarifying or detailing its internal functions~\cite{Arrieta2019}.
	\end{quote}

	We consider \alert{attribution} techniques and \alert{counterfactual} explanations. 

\end{frame}

\begin{frame}{Attribution Techniques}
	\begin{itemize}
		\item Identifies contribution of each input feature wrt. the output
		\item ``Gradient-like'' computations
		\item Based on different propagation rules
	\end{itemize}
	
	\begin{figure}
		\centering
		\includegraphics[width=0.8\textwidth]{explanation_techniques}
		\caption{Direct copy from~\cite{patternnet}.}
	\end{figure}
\end{frame}

\begin{frame}{Counterfactual Explanations}

	Counterfactual explanations tries to answer the question 
	\vspace{1em}

	\begin{quote}
		How can I make a \alert{minimal} and \alert{realistic} change to an input of the model such that the predicted outcome changes?
	\end{quote}
	\vspace{1em}

	\begin{columns}
		\begin{column}{0.5\textwidth}
			\alert{Common strategy}
			\begin{enumerate}
				\item Start from the original input
				\item Until prediction change
					\begin{enumerate}
						\item Make small change to the input
						\item Query the classifier to gain information
					\end{enumerate}
			\end{enumerate}
		\end{column}
		\begin{column}{0.5\textwidth}
			\alert{Differences}
			\begin{itemize}
				\item How to do update
				\item Information gained from classifier
			\end{itemize}
		\end{column}
	\end{columns}

\end{frame}

\begin{frame}[standout]
	Improving Explanations with Probabilistic Saliency Estimation
\end{frame}

\begin{frame}{Probabilistic Saliency Estimation}

\end{frame}

\begin{frame}{Area Under Perturbation Score}

\end{frame}

\begin{frame}[standout]
	Attention Mechanisms in Gradient Explanations
\end{frame}

\begin{frame}{Guiding backward computations}
	A simple dense layer $f_\ell: L' \rightarrow L$ of a Neural Network 
	with backpropagation function $g_\ell: L \rightarrow L'$ of explanation $a_\ell$ is
	\begin{center}
	\begin{tabular}{cc}
		$
			f_\ell(h_{\ell-1}) = \sigma (h_{\ell-1}^T W_\ell)\qquad \qquad
		$ & 
		$
			g_\ell(a_\ell) = a_\ell^T \frac{\partial \sigma(h_{\ell-1}W_\ell)}{\partial h_{\ell-1}w_\ell} W^T
		$ 
	\end{tabular}
	\end{center}
	with $W \in \R^{L' \times  L}$.
	We experimented with guiding the backpropagation
	\begin{align*}
		\hat a_{\ell-1} &= g_\ell(a_\ell) \odot \alpha_\ell(h_{\ell-1}, a_{\ell}) \\ 
    	\alpha_\ell(h_{\ell-1}, a_\ell) &= \tanh \left( [h_{\ell-1} || a_\ell] L_\ell R_\ell \right)
	\end{align*}
	with $L_\ell\in \R^{(L'+L) \times d}$ and $R_\ell \in \R^{d \times L'}$, for $d < \frac{L\cdot L'}{L+L'}$

	\alert{Extensions:}
	\begin{itemize}
		\item Change the input to the $\alpha_\ell$s
		\item Scale the output to be in range $[-c, c]$
	\end{itemize}
\end{frame} 

\begin{frame}{Learning $L$ and $R$}
	\begin{itemize}
		\item Need to definde a loss
		\item DeconvNet and Guided Backprop are class insensitive
	\end{itemize}

	\begin{equation*}
 	 	 L(X, \hat X) = ||\hat X||_1 + \lambda \mathcal{L}(y, \hat y),
	\end{equation*}

	\alert{Extensions:}
	\begin{equation*}
		L(X, \hat X) = \mathcal{L}(y, \hat y) + \lambda_1 ||\hat X||_1 + \lambda_2 ||X - \hat X||_2 + \lambda_3 vec(\hat X)^T \rL vec(\hat X)
	\end{equation*}

	\begin{equation*}
		f^T \rL f = \frac{1}{2}\sum_{i=1}^d \sum_{j=1}^d a_{i,j} (f_i - f_j)^2
	\end{equation*}

\end{frame}

\begin{frame}{Challenges}
	% Loss for attribution functions?
	\begin{itemize}
		\item Currently only potentially works with DeconvNet, GBP, and PatternNet
		\item Loss for attribution methods?
		\item Are these explanations from the in-sample-distribution?
	\end{itemize}

	$$
		\mathrm{AOPC}=\frac{1}{L+1}\left\langle\sum_{k=0}^{L} f\left(x_{\mathrm{MoRF}}^{(0)}\right)-f\left(x_{\mathrm{MoRF}}^{(k)}\right)\right\rangle_{p(x)}
	$$
\end{frame}

\begin{frame}[standout]
	Generating Counterfactual Explanations
\end{frame}


\begin{frame}{Construction}
	$$
	f(x) = \sigma \left ( \beta^T NN(x) \right)
	$$

	$$
	f^\dagger(y) = NN^{-1}\left( f(x) - \frac{f(x)^T \beta}{||\beta||}\right)
	$$
\end{frame}

